\documentclass{ctexart}
\usepackage{amsmath}
\usepackage{amssymb}
\usepackage[overload]{empheq}
\usepackage{enumitem}
\usepackage{lmodern}

\begin{document}

在平面直角坐标系 $xOy$ 中, 水滴曲线由四个 Bézier 控制点描述: \[
    \left(0, 0\right), \left(2s, 0\right), \left(3s, 2s\right), \left(0, 6s\right),
\]
其中 $s>0$ 为尺度参数.

由此写出水滴曲线的参数方程: 
\begin{align*}[left={\empheqlbrace}]
    & x = 3t\left(1-t\right)^2 \cdot 2s + 3t^2\left(1-t\right) \cdot 3s, \\
    & y = 3t^2\left(1-t\right) \cdot 2s + t^3 \cdot 6s,
\end{align*}
其中 $t\in\left[0, 1\right]$.

经化简, 得
\begin{align*}[left={\empheqlbrace}]
    & x = -3st^3-3st^2+6st, \\
    & y = 6st^2.
\end{align*}

以上方程中, 将 $x$ 对 $t$ 求导得 $x'=-9st^2-6st+6s$, 由此可知 $x$ 在 $t\in\left[0, \displaystyle\frac{\sqrt{7}-1}{3}\right]$ 时单调递增, 在 $t\in\left[\displaystyle\frac{\sqrt{7}-1}{3}, 1\right]$ 时单调递减. 故 $x$ 的最大值 $x_{\max} = x\left|_{t=\left(\sqrt{7}-1\right)/3}\right. = \displaystyle\frac{14\sqrt{7}-20}{9}s$.

\mbox{}

将上文的水滴曲线绕 $y$ 轴旋转一周, 即可得到水滴曲面, 其参数方程为
\begin{align*}[left={\empheqlbrace}]
    & x = r\cos{\theta}, \\
    & y = 6st^2, \\
    & z = r\sin{\theta},
\end{align*}
其中 $t\in\left[0, 1\right]$, $r = -3st^3-3st^2+6st$, $\theta\in\left[0, 2\pi\right)$. $r$ 的取值范围为 $\left[0, \displaystyle\frac{14\sqrt{7}-20}{9}s\right]$, $y$ 的取值范围为 $\left[0, 6s\right]$, 由此可用一有限圆柱体包围水滴曲面.

现欲计算某射线是否与水滴曲面相交. 设射线的参数方程为
\begin{align*}[left={\empheqlbrace}]
    & x = q_xl+r_x, \\
    & y = q_yl+r_y, \\
    & z = q_zl+r_z,
\end{align*}
其中 $q_x^2+q_y^2+q_z^2=1$, $r_x, r_y, r_z \in\mathbb{R}$, $l>0$. 分两种情况讨论:
\begin{enumerate}[wide]
    \item $q_y=0$, 此时射线位于平面 $y=r_y$. 若 $r_y\notin\left(0, 6s\right)$, 则\emph{认为}该射线与水滴曲面不相交. 否则, 上述平面与水滴曲面的相交线为平面上的圆 $x^2+z^2 = r_0^2$, 其中 $r_0 = -3st_0^3-3st_0^2+6st_0$, $t_0 = \sqrt{\displaystyle\frac{r_y}{6s}}$, 问题归约为平面上的射线与圆是否相交的问题, 这里不再展开讨论.

    \item $q_y\neq{}0$. 若 $q_y>0$ 且 $r_y\ge{}6s$, 或者 $q_y<0$ 且 $r_y\le{}0$, 则\emph{认为}该射线与水滴曲面不相交. 否则若射线与水滴曲面相交于 $\left(r_0\cos{\theta_0}, 6st_0, r_0\sin{\theta_0}\right)$, 其中 $t_0\in\left[0, 1\right]$, $r_0 = -3st_0^3-3st_0^2+6st_0$, $\theta_0\in\left[0, 2\pi\right)$, 则 $t_0$ 是以下关于 $t$ 的方程的一个根: \[
        \sqrt{\left(q_x\frac{6st^2-r_y}{q_y}+r_x\right)^2 + \left(q_z\frac{6st^2-r_y}{q_y}+r_z\right)^2} = -3st^3-3st^2+6st.
    \]
    若 $q_x=q_z=0$, 则等式左边的值恒为 $\sqrt{r_x^2+r_z^2}$, 令 $u=\sqrt{r_x^2+r_z^2}$. 否则令 $A=\displaystyle\frac{6sq_x}{q_y}$, $B=-\displaystyle\frac{r_yq_x}{q_y}+r_x$, $C=\displaystyle\frac{6sq_z}{q_y}$, $D=-\displaystyle\frac{r_yq_z}{q_y}+r_z$, 等式左边可化为 \[
        \sqrt{\left(A^2+C^2\right)\left(t^2+\frac{AB+CD}{A^2+C^2}\right)^2+\frac{\left(AD-BC\right)^2}{A^2+C^2}},
    \]
    令 $u=\displaystyle\frac{\left|AD-BC\right|}{\sqrt{A^2+C^2}}$. 若 $u \ge \displaystyle\frac{14\sqrt{7}-20}{9}s$, 则\emph{认为}该射线与水滴曲面不相交. 否则解方程 \[
        3st^3+3st^2-6st+\sqrt{r_x^2+r_z^2}=0 \left(q_x=q_z=0\right),
    \] 或 \[
    \begin{aligned}
        & 9s^2t^6+18s^2t^5-\left(A^2+C^2+27s^2\right)t^4-36s^2t^3 \\
        & -2\left(AB+CD-18s^2\right)t^2-\left(B^2+D^2\right)=0 \left(q_x^2+q_z^2>0\right).
    \end{aligned} \]

    对方程的每个根 $t_i$, 求 $l_i = \displaystyle\frac{6st_i^2-r_y}{q_y}$. 若存在 $i$, 使 $t_i\in\left(0, 1\right)$ 且 $l_i>0$, 则射线与水滴曲面有交点 $\left(q_xl_i+r_x, q_yl_i+r_y, q_zl_i+r_z\right)$, 否则\emph{认为}二者无交点.

\end{enumerate}

\end{document}
